\documentclass[12pt,letterpaper,fleqn]{hmcpset}
\usepackage[margin=1in]{geometry}
\usepackage{graphicx}
\usepackage{amsmath,amssymb}
\usepackage{enumerate}
\usepackage{hyperref}
\usepackage{parskip}

% Theorems
\usepackage{amsthm}
\renewcommand\qedsymbol{$\blacksquare$}
\makeatletter
\@ifclassloaded{article}{
    \newtheorem{definition}{Definition}[section]
    \newtheorem{example}{Example}[section]
    \newtheorem{theorem}{Theorem}[section]
    \newtheorem{corollary}{Corollary}[theorem]
    \newtheorem{lemma}{Lemma}[theorem]
}{
}
\makeatother

% Random Stuff
\setlength\unitlength{1mm}

\newcommand{\insertfig}[3]{
\begin{figure}[htbp]\begin{center}\begin{picture}(120,90)
\put(0,-5){\includegraphics[width=12cm,height=9cm,clip=]{#1.eps}}\end{picture}\end{center}
\caption{#2}\label{#3}\end{figure}}

\newcommand{\insertxfig}[4]{
\begin{figure}[htbp]
\begin{center}
\leavevmode \centerline{\resizebox{#4\textwidth}{!}{\input
#1.pstex_t}}
\caption{#2} \label{#3}
\end{center}
\end{figure}}

\long\def\comment#1{}

\newcommand\norm[1]{\left\lVert#1\right\rVert}
\DeclareMathOperator*{\argmin}{arg\,min}
\DeclareMathOperator*{\argmax}{arg\,max}

% bb font symbols
\newfont{\bbb}{msbm10 scaled 700}
\newcommand{\CCC}{\mbox{\bbb C}}

\newfont{\bbf}{msbm10 scaled 1100}
\newcommand{\CC}{\mbox{\bbf C}}
\newcommand{\PP}{\mbox{\bbf P}}
\newcommand{\RR}{\mbox{\bbf R}}
\newcommand{\QQ}{\mbox{\bbf Q}}
\newcommand{\ZZ}{\mbox{\bbf Z}}
\renewcommand{\SS}{\mbox{\bbf S}}
\newcommand{\FF}{\mbox{\bbf F}}
\newcommand{\GG}{\mbox{\bbf G}}
\newcommand{\EE}{\mbox{\bbf E}}
\newcommand{\NN}{\mbox{\bbf N}}
\newcommand{\KK}{\mbox{\bbf K}}
\newcommand{\KL}{\mbox{\bbf KL}}

% Vectors
\renewcommand{\aa}{{\bf a}}
\newcommand{\bb}{{\bf b}}
\newcommand{\cc}{{\bf c}}
\newcommand{\dd}{{\bf d}}
\newcommand{\ee}{{\bf e}}
\newcommand{\ff}{{\bf f}}
\renewcommand{\gg}{{\bf g}}
\newcommand{\hh}{{\bf h}}
\newcommand{\ii}{{\bf i}}
\newcommand{\jj}{{\bf j}}
\newcommand{\kk}{{\bf k}}
\renewcommand{\ll}{{\bf l}}
\newcommand{\mm}{{\bf m}}
\newcommand{\nn}{{\bf n}}
\newcommand{\oo}{{\bf o}}
\newcommand{\pp}{{\bf p}}
\newcommand{\qq}{{\bf q}}
\newcommand{\rr}{{\bf r}}
\renewcommand{\ss}{{\bf s}}
\renewcommand{\tt}{{\bf t}}
\newcommand{\uu}{{\bf u}}
\newcommand{\ww}{{\bf w}}
\newcommand{\vv}{{\bf v}}
\newcommand{\xx}{{\bf x}}
\newcommand{\yy}{{\bf y}}
\newcommand{\zz}{{\bf z}}
\newcommand{\0}{{\bf 0}}
\newcommand{\1}{{\bf 1}}

% Matrices
\newcommand{\Ab}{{\bf A}}
\newcommand{\Bb}{{\bf B}}
\newcommand{\Cb}{{\bf C}}
\newcommand{\Db}{{\bf D}}
\newcommand{\Eb}{{\bf E}}
\newcommand{\Fb}{{\bf F}}
\newcommand{\Gb}{{\bf G}}
\newcommand{\Hb}{{\bf H}}
\newcommand{\Ib}{{\bf I}}
\newcommand{\Jb}{{\bf J}}
\newcommand{\Kb}{{\bf K}}
\newcommand{\Lb}{{\bf L}}
\newcommand{\Mb}{{\bf M}}
\newcommand{\Nb}{{\bf N}}
\newcommand{\Ob}{{\bf O}}
\newcommand{\Pb}{{\bf P}}
\newcommand{\Qb}{{\bf Q}}
\newcommand{\Rb}{{\bf R}}
\newcommand{\Sb}{{\bf S}}
\newcommand{\Tb}{{\bf T}}
\newcommand{\Ub}{{\bf U}}
\newcommand{\Wb}{{\bf W}}
\newcommand{\Vb}{{\bf V}}
\newcommand{\Xb}{{\bf X}}
\newcommand{\Yb}{{\bf Y}}
\newcommand{\Zb}{{\bf Z}}

% Calligraphic
\newcommand{\Ac}{{\cal A}}
\newcommand{\Bc}{{\cal B}}
\newcommand{\Cc}{{\cal C}}
\newcommand{\Dc}{{\cal D}}
\newcommand{\Ec}{{\cal E}}
\newcommand{\Fc}{{\cal F}}
\newcommand{\Gc}{{\cal G}}
\newcommand{\Hc}{{\cal H}}
\newcommand{\Ic}{{\cal I}}
\newcommand{\Jc}{{\cal J}}
\newcommand{\Kc}{{\cal K}}
\newcommand{\Lc}{{\cal L}}
\newcommand{\Mc}{{\cal M}}
\newcommand{\Nc}{{\cal N}}
\newcommand{\Oc}{{\cal O}}
\newcommand{\Pc}{{\cal P}}
\newcommand{\Qc}{{\cal Q}}
\newcommand{\Rc}{{\cal R}}
\newcommand{\Sc}{{\cal S}}
\newcommand{\Tc}{{\cal T}}
\newcommand{\Uc}{{\cal U}}
\newcommand{\Wc}{{\cal W}}
\newcommand{\Vc}{{\cal V}}
\newcommand{\Xc}{{\cal X}}
\newcommand{\Yc}{{\cal Y}}
\newcommand{\Zc}{{\cal Z}}

% Bold greek letters
\newcommand{\alphab}{\hbox{\boldmath$\alpha$}}
\newcommand{\betab}{\hbox{\boldmath$\beta$}}
\newcommand{\gammab}{\hbox{\boldmath$\gamma$}}
\newcommand{\deltab}{\hbox{\boldmath$\delta$}}
\newcommand{\etab}{\hbox{\boldmath$\eta$}}
\newcommand{\lambdab}{\hbox{\boldmath$\lambda$}}
\newcommand{\epsilonb}{\hbox{\boldmath$\epsilon$}}
\newcommand{\nub}{\hbox{\boldmath$\nu$}}
\newcommand{\mub}{\hbox{\boldmath$\mu$}}
\newcommand{\zetab}{\hbox{\boldmath$\zeta$}}
\newcommand{\phib}{\hbox{\boldmath$\phi$}}
\newcommand{\psib}{\hbox{\boldmath$\psi$}}
\newcommand{\thetab}{\hbox{\boldmath$\theta$}}
\newcommand{\taub}{\hbox{\boldmath$\tau$}}
\newcommand{\omegab}{\hbox{\boldmath$\omega$}}
\newcommand{\xib}{\hbox{\boldmath$\xi$}}
\newcommand{\sigmab}{\hbox{\boldmath$\sigma$}}
\newcommand{\pib}{\hbox{\boldmath$\pi$}}
\newcommand{\rhob}{\hbox{\boldmath$\rho$}}

\newcommand{\Gammab}{\hbox{\boldmath$\Gamma$}}
\newcommand{\Lambdab}{\hbox{\boldmath$\Lambda$}}
\newcommand{\Deltab}{\hbox{\boldmath$\Delta$}}
\newcommand{\Sigmab}{\hbox{\boldmath$\Sigma$}}
\newcommand{\Phib}{\hbox{\boldmath$\Phi$}}
\newcommand{\Pib}{\hbox{\boldmath$\Pi$}}
\newcommand{\Psib}{\hbox{\boldmath$\Psi$}}
\newcommand{\Thetab}{\hbox{\boldmath$\Theta$}}
\newcommand{\Omegab}{\hbox{\boldmath$\Omega$}}
\newcommand{\Xib}{\hbox{\boldmath$\Xi$}}

% mixed symbols
\newcommand{\sinc}{{\hbox{sinc}}}
\newcommand{\diag}{{\hbox{diag}}}
\renewcommand{\det}{{\hbox{det}}}
\newcommand{\trace}{{\hbox{tr}}}
\newcommand{\tr}{\trace}
\newcommand{\sign}{{\hbox{sign}}}
\renewcommand{\arg}{{\hbox{arg}}}
\newcommand{\var}{{\hbox{var}}}
\newcommand{\cov}{{\hbox{cov}}}
\renewcommand{\Re}{{\rm Re}}
\renewcommand{\Im}{{\rm Im}}
\newcommand{\eqdef}{\stackrel{\Delta}{=}}
\newcommand{\defines}{{\,\,\stackrel{\scriptscriptstyle \bigtriangleup}{=}\,\,}}
\newcommand{\<}{\left\langle}
\renewcommand{\>}{\right\rangle}
\newcommand{\Psf}{{\sf P}}
\newcommand{\T}{\top}
\newcommand{\m}[1]{\begin{bmatrix} #1 \end{bmatrix}}


% info for header block in upper right hand corner
\name{}
\class{Math189R SP19}
\assignment{Homework 3}
\duedate{Monday, Feb 18, 2019}

\begin{document}
\newcommand{\sumk}{\sum_{i=1}^K}
\newcommand{\sumkL}{\sum_{i=1}^{K-1}}
\newcommand{\eeta}{\boldsymbol{\eta}}
\newcommand{\pro}{\PP(\theta;a,b)}
Feel free to work with other students, but make sure you write up the homework
and code on your own (no copying homework \textit{or} code; no pair programming).
Feel free to ask students or instructors for help debugging code or whatever else,
though.

\begin{problem}[1]
(\textbf{Murphy 2.16}) Suppose $\theta \sim \text{Beta}(a,b)$ such
        that
        \[
            \PP(\theta; a,b) = \frac{1}{B(a,b)} \theta^{a-1}(1-\theta)^{b-1} = \frac{\Gamma(a+b)}{\Gamma(a)\Gamma(b)} \theta^{a-1}(1-\theta)^{b-1}
        \]
        where $B(a,b) = \Gamma(a)\Gamma(b)/\Gamma(a+b)$ is the Beta function
        and $\Gamma(x)$ is the Gamma function.
        Derive the mean, mode, and variance of $\theta$.
\end{problem}
\begin{solution}
\begin{enumerate}
\item[Mean] \textit{The mean value is the expected value; }
\begin{align*}
\EE(\theta) &= \int \theta \pro d\theta = \int \theta \frac{1}{B(a,b)} \theta^{a-1} (1-\theta)^{b-1} d\theta\\
&= \frac{1}{B(a,b)} \int \theta^{a}(1-\theta)^{b-1} d\theta \\
&= \frac{B(a+1,b)}{B(a,b)}\\
&= \left[\frac{\Gamma(a+1)\Gamma(b)}{\Gamma(a+b+1)}\right] \left[\frac{\Gamma(a+b)}{\Gamma(a)\Gamma(b)}\right]\\
&=\left[\frac{a\Gamma(a)\Gamma(b)}{a+b\Gamma(a+b)}\right] \left[\frac{\Gamma(a+b)}{\Gamma(a)\Gamma(b)}\right]\\
&=\boxed{\frac{a}{a+b}}
\end{align*}
\item[Variance] \textit{We know that this is $\EE[\theta^2]-\EE[\theta]^2$;}
\begin{align*}
\EE[\theta^2]-\EE[\theta]^2&= \int \theta^2 \pro d\theta - \frac{a^2}{(a+b)^2}\\
&=\frac{1}{B(a,b)}\int \theta^{a+1}(1-\theta)^{b-1}d \theta  - \frac{a^2}{(a+b)^2}\\
&= \frac{B(a+2,b)}{B(a,b)} - \frac{a^2}{(a+b)^2}\\
&= \frac{\Gamma(a+2)\Gamma(b)}{\Gamma(a+b+2)}\frac{\Gamma(a+b)}{\Gamma(a)\Gamma(b)}- \frac{a^2}{(a+b)^2}\\
&= \frac{a(a+1)\Gamma(a)\Gamma(b)}{(a+b)(a+b+1)\Gamma(a+b)}\frac{\Gamma(a+b)}{\Gamma(a)\Gamma(b)}- \frac{a^2}{(a+b)^2}\\
&=\frac{a(a+1)}{(a+b)(a+b+1)}- \frac{a^2}{(a+b)^2}\\
&=\frac{(a^2+a)(a+b)-a^2(a+b+1)}{(a+b+1)(a+b)^2}\\
&=\boxed{\frac{ab}{(a+b+1)(a+b)^2}}
\end{align*}
\item[Mode] \textit{If we visualize a distribution, the point at which there is the greatest number of values is the mode, it is also the only global extrema, as such, it is the only point where$\nabla_{\theta} = 0$;}
\begin{align*}
0 &= \nabla_{\theta}\left(\frac{1}{B(a,b)}\theta^{a-1}(1-\theta)^{b-1}\right)\\
 &=\nabla_{\theta}(\theta^{a-1}(1-\theta)^{b-1}\\
 &=(a-1)\theta^{a-2}(1-\theta)^{b-1}-(b-1)\theta^{a-1}(1-\theta)^{b-2}\\
\end{align*}
\textit{Therefore we can now solve for the specific value of $\theta$ that makes this true; }
\begin{align*}
(a-1)\theta^{a-2}(1-b)^{b-1} &= (b-1)\theta^{a-1}(1-\theta)^{b-2}\\
(a-1)(1-\theta)&=(b-1)\theta\\
\theta &=\boxed{ \frac{a-1}{a+b-2}}
\end{align*}
\end{enumerate}
\end{solution}
\newpage

\begin{problem}[2]
(\textbf{Murphy 9}) Show that the multinoulli distribution
\[
    \text{Cat}(\xx|\mub) = \prod_{i=1}^K \mu_i^{x_i}
\]
is in the exponential family and show that the generalized linear model
corresponding to this distribution is the same as multinoulli logistic
regression (softmax regression).
\end{problem}
\begin{solution}
We are searching for something of the form $\pp(\yy;\boldsymbol{\eta}=b(\eeta)\exp \left(\eeta^TT(\yy)+a(\eeta)\right)$, as we did in class I am going to force the multinoulli distribution $\prod_{i=1}^K \mu_i^{x_i}$ into its exponential form:
\begin{align}
\prod_{i=1}^K \mu_i^{x_i} &= \exp\left[\log\left(\prod_{i=1}^K \mu_i^{x_i}\right)\right]\\
&=\exp\left[\sumk\log(\mu_i^{x_i})\right]\\
&= \exp \left[\sumk x_i\log(\mu_i)\right]
\end{align}
Now lets note that since this is a multinoulli distribution; 
$$\mu_k = 1-\sumkL\mu_i \text{ and } x_k = 1- \sumkL x_i$$
\begin{align}
\phantom{\prod_{i=1}^K \mu_i^{x_i} }&= \exp\left[\sumkL x_i\log(\mu_i)+\left(1-\sumkL x_i\right)\log (\mu_k)\right]\\
&=\exp\left[\sumkL x_i \log\left(\frac{\mu_i}{\mu_k}\right)+\log (\mu_k)\right]
\end{align}
Lets continue by defining $\eeta$ as follows; 
$$ \eeta = \begin{pmatrix}
\log\left( \frac{\mu_1}{\mu_k}\right)\\
\vdots\\
\log\left( \frac{\mu_{k-1}}{\mu_k}\right)
\end{pmatrix}
$$
It follows that we can define $\mu_k$ as $1-\sumkL\mu_k\exp(\eta_i) $ or equivalently as $ \frac{1}{1 +\sumkL exp(\eta_i)} $
Thus picking up where we left off; 
\begin{align}
\phantom{\prod_{i=1}^K \mu_i^{x_i} }&= \exp\left[\eeta^T\xx + \log\left(\frac{1}{1 +\sumkL \exp(\eta_i)}\right)\right]
\end{align}
This is in exponential form with $b(y)  =1 $ , $T(y) = \xx$ and $a(\eeta) = -\log( 1 +\sumkL \exp(\eta_i))$
\end{solution}
\newpage

\end{document}
